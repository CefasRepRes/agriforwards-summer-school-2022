\documentclass[conference]{IEEEtran}
\IEEEoverridecommandlockouts
% The preceding line is only needed to identify funding in the first footnote. If that is unneeded, please comment it out.
\usepackage{cite}
\usepackage{amsmath,amssymb,amsfonts}
\usepackage{algorithmic}
\usepackage{graphicx}
\usepackage{textcomp}
\usepackage{xcolor}
\def\BibTeX{{\rm B\kern-.05em{\sc i\kern-.025em b}\kern-.08em
    T\kern-.1667em\lower.7ex\hbox{E}\kern-.125emX}}
\begin{document}

\title{The prospects for robot fishing\\
{\footnotesize \textsuperscript{*}Note: Sub-titles are not captured in Xplore and
should not be used}
\thanks{$^1$This work was supported by the Engineering and Physical Sciences Research Council [EP/S023917/1]}
}

\author{\IEEEauthorblockN{1\textsuperscript{st} Given Name Surname}
\IEEEauthorblockA{\textit{dept. name of organization (of Aff.)} \\
\textit{name of organization (of Aff.)}\\
City, Country \\
email address}
\and
\IEEEauthorblockN{2\textsuperscript{nd} Garry Clawson$^1$}
\IEEEauthorblockA{\textit{School of Computer Science,} \\
\textit{University of Lincoln}, UK \\
18685030@students.lincoln.ac.uk}
\and
\IEEEauthorblockN{3\textsuperscript{rd} Given Name Surname}
\IEEEauthorblockA{\textit{dept. name of organization (of Aff.)} \\
\textit{name of organization (of Aff.)}\\
City, Country \\
email address}
%\and
%\IEEEauthorblockN{4\textsuperscript{th} Given Name Surname}
%\IEEEauthorblockA{\textit{dept. name of organization (of Aff.)} \\
%\textit{name of organization (of Aff.)}\\
%City, Country \\
%email address}
%\and
%\IEEEauthorblockN{5\textsuperscript{th} Given Name Surname}
%\IEEEauthorblockA{\textit{dept. name of organization (of Aff.)} \\
%\textit{name of organization (of Aff.)}\\
%City, Country \\
%email address}
%\and
%\IEEEauthorblockN{6\textsuperscript{th} Given Name Surname}
%\IEEEauthorblockA{\textit{dept. name of organization (of Aff.)} \\
%\textit{name of organization (of Aff.)}\\
%City, Country \\
%email address}
}

\maketitle

\begin{abstract}
Abstract goes here
\end{abstract}

\begin{IEEEkeywords}
fishing, robotics
\end{IEEEkeywords}

\section{Introduction}

In July 2022, 28 AgriFoRwArdS students participated in a summer school
at the University of East Anglia where they were challenged to
imagine, design, prototype and cost a robot system for catching fish.

AgriForWardS is a Centre for Doctoral Training (CDT) in Agri-Food
Robotics established by the University of Lincoln in collaboration
with the University of Cambridge and University of East Anglia.

    \subsection{Contribution of paper}
Technological trends within the fishing industry are not well documented nor well defined. 
This survey aims to make such identifications and begins by reviewing the specific needs of the fishing industry.
It then discusses the current regulatory framework that fishers operate within.
Each stage of the fisher process is then introduced form finding fish, catching, sorting and preparing. 

\section{The fishing industry}
[GC sub section ideas:

1. regulatory framework

2. current catch and landing trends of UK fishing landscape (fish sellers)

3. Current technologies \cite{b4}

4. Current technology limitations

5. safety in fishing

6. financial considerations

7. drive towards sustainability



\section{Concept}

\section{The fish finder}
Example papers: \cite{b7} \cite{b8} \cite{b9} \cite{b10}

\section{The fish catcher}
Example papers:


\section{The fish sorter}
Example papers: \cite{b1} \cite{b2} \cite{b3} \cite{b5} \cite{b6}


\section{The fish cooker}
Example papers:

\section{Discussion}

\section{Conclusions}

\begin{thebibliography}{00}
\bibitem{b1} Bonifácio, Lucas T., et al. "Explainable Classification Methods for Fish Species Detection Using Hydroacoustic Data." 2021 IEEE International Conference on Fuzzy Systems (FUZZ-IEEE). IEEE, 2021.

\bibitem{b2} Lefort, Riwal, et al. "Spatial statistics of objects in 3-D sonar images: application to fisheries acoustics." IEEE Geoscience and Remote Sensing Letters 9.1 (2011): 56-59.

\bibitem{b3} Marques, Tunai Porto, et al. "Detecting marine species in echograms via traditional, hybrid, and deep learning frameworks." 2020 25th International Conference on Pattern Recognition (ICPR). IEEE, 2021.

\bibitem{b4} Gladju, J., Biju Sam Kamalam, and A. Kanagaraj. "Applications of data mining and machine learning framework in aquaculture and fisheries: A review." Smart Agricultural Technology (2022): 100061.

\bibitem{b5} Marques, Tunai Porto, et al. "Detecting marine species in echograms via traditional, hybrid, and deep learning frameworks." 2020 25th International Conference on Pattern Recognition (ICPR). IEEE, 2021.

\bibitem{b6} Schmidt, Marc B., Jeffrey A. Tuhtan, and Martin Schletterer. "Hydroacoustic and pressure turbulence analysis for the assessment of fish presence and behavior upstream of a vertical trash rack at a run-of-river hydropower plant." Applied Sciences 8.10 (2018): 1723.

\bibitem{b7} Cui, Suxia, et al. "Fish detection using deep learning." Applied Computational Intelligence and Soft Computing 2020 (2020).

\bibitem{b8} Mohamed, Hussam El-Din, et al. "Msr-yolo: Method to enhance fish detection and tracking in fish farms." Procedia Computer Science 170 (2020): 539-546.

\bibitem{b9} Jalal, Ahsan, et al. "Fish detection and species classification in underwater environments using deep learning with temporal information." Ecological Informatics 57 (2020): 101088.

\bibitem{b10} Salman, Ahmad, et al. "Automatic fish detection in underwater videos by a deep neural network-based hybrid motion learning system." ICES Journal of Marine Science 77.4 (2020): 1295-1307.

\bibitem{b11} Liu, Shasha, et al. "Embedded online fish detection and tracking system via YOLOv3 and parallel correlation filter." OCEANS 2018 MTS/IEEE Charleston. IEEE, 2018.









\end{thebibliography}
\end{document}
